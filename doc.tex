\documentclass{article}
%\usepackage{hyperlatex}
\usepackage{fullpage}

\title{Scala Actors Documentation}
\author{Stephen Tu}

\begin{document}

\maketitle

\section{Remote Actors}
This portion describes the newly redesigned remote actors API (since 2.8).
The API was designed to be fully source compatible, meaning existing remote
actor programs written should still work (with the same semantics). Note however
that the redesigned network protocol is not backwards compatible with the previous
network protocol.

\subsection{RemoteActor}
The main interface to the remote actor API is the \verb|scala.actos.remote.RemoteActor| object.
This object provides the necessary means to create and connect to remote actor instances.

\subsubsection{Starting Remote Actors}
A remote actor is uniquely identified by a \verb|Symbol|. This symbol is unique to the 
JVM instance. The two idioms to create a remote actor identified with name \verb|myActor|
are as follows. Note for all code snippets shown below, assume the following imports:
\begin{verbatim}
import scala.actors._
import scala.actors.Actor._
import scala.actors.remote._
import scala.actors.remote.RemoteActor._
\end{verbatim}
The anonymous syntax for creation is similiar to the \verb|actor| syntax:
\begin{verbatim}
val myActor = remoteActor(9000, 'myActor) {
  loop {
    react {
      case e => 
        sender ! e // an echo actor
    }
  }
}
\end{verbatim}
Explicit actor classes can be made remote as follows (this is also shown in the Scaladoc):
\begin{verbatim}
// same imports as before
class MyActor extends Actor {
  def act() {
    alive(9000)
    register(self, 'myActor)
    // actor behavior
  }
}
\end{verbatim}
Note that a name can only be registered to a single (alive) actor at a time.
For example, if one wants to register actor $A$ as \verb|myActor|, and then register
another actor $B$ as \verb|myActor|, one would have to wait until $A$ terminated,
or explicitly call \verb|unregister|, passing in actor $A$. This requirement
applies across all ports, so simply registering $B$ on a different port as $A$
is not sufficient.

\subsubsection{Connecting to Remote Actors}
Connecting to a remote actor created by the means described previously is also
just as simple. \verb|RemoteActor| defines a type alias called \verb|RemoteProxy|. 
This is a representation of a remote actor handle, which is returned by calls to
various methods in \verb|RemoteActor|, including \verb|select|. To connect
to a remote actor running on machine with address \verb|myMachine|, on port \verb|8000|,
with name \verb|aActor|, use \verb|select|\footnote{
  Note that \texttt{select} should be called within an actor. To select a remote actor
  outside of an actor, use \texttt{remoteActorAt}, which has a method signature
  identical to \texttt{select}. The difference between \texttt{select} and \texttt{remoteActorAt}
  is that \texttt{select} uses the thread local \texttt{scala.actors.Actor.self} to 
  register a termination handler in the network kernel, so that when all actors shut down,
  the network kernel can release its internel thread pools, and the JVM can exit properly.
  However, if called outside of an actor, \texttt{Actor.self} returns an actor proxy for the
  current thread, which never invokes its termination handlers. Consequently, the network
  kernel never gets notified when the thread terminates, does not release its resources, and
  the JVM keeps on running. \texttt{remoteActorAt} gets around this by simply not registering
  any termination handlers, so the current thread does not prevent the network kernel
  from shutting down.
}
in the following manner:
\begin{verbatim}
val aActor = select(Node("myMachine", 8000), 'aActor)
\end{verbatim}
The returned actor from \verb|select| has the same interface as a regular actor, and thus
can be used as such:
\begin{verbatim}
aActor ! "Hello, aActor!"
receive {
  case response => println("Response: " + response)
}
aActor !? "What is the meaning of life?" match {
  case 42   => println("Success")
  case oops => println("Failed: " + oops)
}
val ftch = aActor !! "What is the last digit of PI?"
// stash ftch away to be invoked (much) later
\end{verbatim}
Note that both \verb|select| and \verb|remoteActorAt| are lazy; they do not actually initiate any
network connections. They simple create a new \verb|RemoteActor| instance which is ready
to initiate a new network connection when needed (for instance, when \verb|!| is invoked). 
When a new connection is initiated, there are three potential sources of error:
\begin{enumerate}
  \item The TCP connection cannot be made (bad address, bad port, etc)
  \item The remote end does not speak the remote actors protocol, 
  \item The remote actor named does not exist remotely
\end{enumerate}
The current implementation only handles the first and last error cases gracefully, by 
throwing exceptions. The middle case causes undefined behavior.
Exception handling for remote actors will be covered in a later section, 
since there are a few subtleties to be aware of.

\subsection{Configuration}
One of the major changes of the remote actors API is the \verb|Configuration| object.
The method signature for many of the methods in \verb|RemoteActor| take an
implicit \verb|Configuration| instance. For instance, consider the signature
for \verb|alive|:
\begin{verbatim}
def alive(port: Int, actor: Actor)(implicit cfg: Configuration)
\end{verbatim}
The default \verb|Configuration| instance used when no explicit instance is specified
is \\ \verb|scala.actors.remote.Configuration.DefaultConfig|. By looking at the trait
definition for \verb|Configuration|, one sees many parameters open for configuration.
\begin{enumerate}
  \item \verb|aliveMode| - The \verb|ServiceMode| that listeners should use
  \item \verb|selectMode| - The \verb|ServiceMode| that \verb|RemoteActor| instances should use
  \item \verb|lookupValidPeriod| - The time (in milliseconds) to cache a lookup for a particular 
    remote actor on a machine.
  \item \verb|waitTimeout| - How long the network kernel should wait up to for any blocking operations.
    Right now, this is only used by the various \verb|remoteStart| methods to determine how long
    to wait up to for a response.
  \item \verb|newSerializer| - A new \verb|Serializer| instance to use for a new connection. 
    Read the section on \verb|Serializer| instances to understand how this is used. The
    default is Java serialization.
  \item \verb|newMessageCreator| - A new \verb|MessageCreator| instance to use for a new connection.
    Read the section on \verb|Serializer| instances to understand how this is used.
  \item \verb|numRetries| - How many times a particular \verb|RemoteActor| should retry to deliver a message
    until it determines the endpoint is not reachable and throws an error.
  \item \verb|classLoader| - The class loader to use whenever a new instance of a class is needed (via reflection). 
    Currently only used by the network kernel to create a new actor (via \verb|remoteStart|).
\end{enumerate}
It is simple to use a custom \verb|Configuration| instance via implicit values:
\begin{verbatim}
implicit val mySpecialConfig = new DefaultConfiguration {
  override val aliveMode = ServiceMode.NonBlocking
  override val selectMode = ServiceMode.NonBlocking 
}
val fooActor = select(Node(9000), 'fooActor) // uses mySpecialConfig via an implicit parameter
\end{verbatim}
Most of the options have reasonable default values, but the two that do not are \verb|aliveMode| and
\verb|selectMode|. 

\subsubsection{Service Modes}
What exactly is a service mode? It is an enumeration which takes on two values (\verb|Blocking| and \verb|NonBlocking|),
and defines which mode the network kernel should operate in. \verb|Blocking| corresponds to 
using a traditional one thread per connection model, backed by implementations found in 
\verb|java.io.*|. \verb|NonBlocking| corresponds to using a single thread\footnote{
  While in theory a single thread is sufficient for NIO, the actual implementation
  maintains a thread pool of size $2N_{cpu}$, where each thread runs a selector. Incoming
  connections are then assigned in round-robin fashion a selector thread. This in theory
  allows greater scalability. %TODO: cite some blog post about this
}
to multiplex IO channels, backed by implementations found in \verb|java.nio.*|. The mode that
one uses depends on the application at hand, and only proper profiling can really answer the
question. As a general rule of thumb however, \verb|Blocking| mode is better suited for 
lower traffic applications (where the number of connections between nodes is not expected
to exceed $1000$), and \verb|NonBlocking| mode is better suited when a large number of
concurrent connections between nodes is expected. Additionally, even though the option of placing
both \verb|aliveMode| and \verb|selectMode| in \verb|NonBlocking| mode is given, in most 
cases when \verb|NonBlocking| is desirable, it really means \verb|NonBlocking| is desirable for
the \verb|aliveMode|, since running client connections in NIO mode does not really give as many
of the benefits as running a server socket in NIO mode.

\subsection{Serializer}
Another new addition to the remote actors API is the ability to use custom serializers
to handle the serialization of messages into byte streams. By default, a serializer which
uses Java serialization is used, because this provides the most flexibility (all primitives work,
Strings, Lists, Tuples, etc). However, Java serialization is known to be slower than most
specialized serialization frameworks on the JVM\footnote{\texttt{http://wiki.github.com/eishay/jvm-serializers}}
(since it tries to reconstruct the object graph as accurately as possible). Thus, if you want
to take advantage of one of these faster serialization frameworks, then defining a
custom \verb|Serializer| is the way to go. Doing so, however, requires a proper understanding
of the remote actor network protocol.

\subsubsection{Serializer Identification}
Each serializer must supply a \verb|uniqueId| value, which is intended to uniquely identify the
serializer on the network. The reason for this is, since serializers can be swapped out, when
a connection is established, one wants to be sure with reasonable confidence that the other
side can understand the byte streams being sent, and vice versa. Thus, serializers all contain
a \verb|uniqueId| field, which is sent in the default handshake protocol and compared.

\subsubsection{Handshaking}
Each network connection is tied to a serializer instance. 
When the network kernel establishes a TCP connection, it calls back to the serializer to ask it if
it wants to engage in a handshake with the other side. This is determined by the
\verb|isHandshaking| value. If this value is set to \verb|true|, then the
network kernel repeatedly invokes its \verb|handleNextEvent| method to engage in a handshake
with the remote end. The method signature for \verb|handleNextEvent| is given as:
\begin{verbatim}
def handleNextEvent: PartialFunction[ReceivableEvent, Option[TriggerableEvent]]
\end{verbatim}
The Scaladoc for \verb|handleNextEvent| explains nicely how the method is invoked.
The idea here is that two sides of the connection exchange primitive messages back and forth
to set up some agreement on how to communicate. By default, this handshake simply exchanges
IDs to make sure they are equal. This is implemented as follows, as a reusable mix-in:
\begin{verbatim}
trait IdResolvingSerializer { _: Serializer =>
  override val isHandshaking = true        
  override def handleNextEvent: PartialFunction[ReceivableEvent, Option[TriggerableEvent]] = {                        
    case StartEvent(_)                     => Some(SendEvent(uniqueId))
    case RecvEvent(id) if (id == uniqueId) => Some(Success)                                                           
    case RecvEvent(id)                     => 
      Some(Error("ID's do not match. Expected " + uniqueId + ", but got " + id))     
  }                                                                                                                   
}
\end{verbatim}
More complicated handshakes can be implemented. For instance, it is possible to
express a schema resolution handshake in this framework, so that when both sides
have slightly differing views of the world (in terms of message schemas), they 
can negotiate a translation mechanism, instead of just failing. Note that,
messages that can be passed remotely during a handshake are restricted to primitives,
strings, and byte arrays. This error is only detected at runtime, however. The
reason for this is because usually when serializers are handshaking, it is because
they are not aware of how to communicate with each other- thus a pre-defined bootstrap
format must be used, which is defined in \verb|PrimitiveSerializer|. \\ \\
If a handshake fails, that is, if an \verb|Error| event is returned from
an invocation to \verb|handleNextEvent|, the network kernel propagates this up the
call stack as an exception.

\subsubsection{Wire Format}
Once the handshake process has succeeded, serializers then proceed to communicate
via a predefined wire format. Specifically, there are a fixed set of first-class
command messages that can be sent from a node to another node. These messages are:
\begin{enumerate}
  \item \verb|LocateRequest| - Sent to inquire the remote end if an actor with a specific
    name exists.
  \item \verb|LocateResponse| - A response to a \verb|LocateRequest|.
  \item \verb|AsyncSend| - Sent for an asynchronous message, that is, any message
    sent via \verb|!|.
  \item \verb|SyncSend| - Sent for a synchronous message, that is, any message sent
    via \verb|!!| or \verb|!?|.
  \item \verb|SyncReply| - A response to a \verb|SyncSend|.
  \item \verb|RemoteApply| - Used to implement \verb|link| and \verb|unlink|.
\end{enumerate}
The \verb|Serializer| base class has an abstract method \verb|writeMessageName| for each 
of these message types. For instance, for \verb|LocateRequest|:
\begin{verbatim}
def writeLocateRequest(outputStream: OutputStream, sessionId: Long, receiverName: String): Unit
\end{verbatim}
There is also an abstract \verb|read| method which has the following signature:
\begin{verbatim}
def read(inputStream: InputStream): NetKernelMessage
\end{verbatim}
Thus, this suggests that a \verb|Serializer| is free to choose any binary format to write the message,
and as long as the \verb|read| method can decode the binary format and somehow produce a 
\verb|NetKernelMessage| instance, then the network kernel does not care how this happened.
This is exactly the case, and is done intentionally to give writers of \verb|Serializer|s the most
flexibility.\\ \\
The default \verb|JavaSerializer| implements this by first writing a tag (byte) which is associated
with each of the messages, followed by the meta data fields, and then the message.

\subsubsection{An Example: GZip Serializer}
To make this more concrete, here is an example of a \verb|Serializer| which wraps an underlying
\verb|Serializer| and does GZip compression on the byte streams. For the GZip compression,
we will simply use the one defined in \verb|java.util.zip.GZIPOutputStream|.
\begin{verbatim}
class GZipSerializer(val underlying: Serializer) extends Serializer {
  override val uniqueId = 228081497L
  override val isHandshaking = true

  private var node: Node = _
  private var underlyingStarted = false

  type HandshakePF = PartialFunction[ReceivableEvent, Option[TriggerableEvent]]

  override def handleNextEvent: HandshakePF = {
    case StartEvent(n) =>
      node = n
      Some(SendEvent(uniqueId))
    case RecvEvent(msg) if (!underlyingStarted) =>
      if (msg != uniqueId)
        Some(Error("Did not send expected uniqueId"))
      else
        underlyingStarted = true
        underlying.handleNextEvent(StartEvent(node))
    case e if (underlyingStarted) =>
      underlying.handleNextEvent(e)
  }

  override def writeLocateRequest(outputStream: OutputStream, 
      sessionId: Long, receiverName: String) {
    val gos = new GZIPOutputStream(outputStream)
    underlying.writeLocateRequest(gos, sessionId, receiverName)
    gos.finish()
  }

  // very similar write methods omitted for the remaining message types

  override def read(bytes: InputStream) =
    underlying.read(new GZIPInputStream(bytes))
}
\end{verbatim}



\end{document}

